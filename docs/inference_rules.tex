\documentclass{article}
\usepackage{bussproofs}
\usepackage{amsfonts,amsmath,amssymb,amsthm}
\usepackage{amsmath}
\usepackage{marvosym}
\usepackage{unicode-math}
\usepackage[margin=0.5in]{geometry}
\usepackage{newunicodechar}
\usepackage{fancyvrb}
\AtBeginDocument{\setmainfont{XITS-Regular.otf}}
\AtBeginDocument{\setmathfont{XITSMath-Regular.otf}}
\AtBeginDocument{\newfontfamily{\mathfont}{FreeMono.otf}}


\AtBeginDocument{\newunicodechar{→} {\mathfont{→}}}
\AtBeginDocument{\newunicodechar{∀} {\ensuremath{\forall}}}
\AtBeginDocument{\newunicodechar{∃} {\ensuremath{\exists}}}
\AtBeginDocument{\newunicodechar{‣} {\mathfont{‣}}}
\AtBeginDocument{\newunicodechar{∴} {\mathfont{∴}}}
\AtBeginDocument{\newunicodechar{≔} {\mathfont{≔}}}
\AtBeginDocument{\newunicodechar{⋮} {\mathfont{⋮}}}
\AtBeginDocument{\newunicodechar{∧} {\mathfont{∧}}}
\AtBeginDocument{\newunicodechar{∨} {\mathfont{∨}}}
\AtBeginDocument{\newunicodechar{⚡} {\mbox{\Lightning}}}

\title{"Chowder" proof checking inference rules augmented with proposed syntax}
\author{matt rice}

\newenvironment{bprooftree0}
  {\leavevmode\hbox\bgroup}
  {\DisplayProof\egroup}
\newenvironment{bprooftree}
  {\noindent\hbox\bgroup}
  {\DisplayProof\egroup}

\newcommand{\charge}[2]{%
    \LeftLabel{$\smalltriangleright$}
    \RightLabel{$^+\text{\Lightning}\alpha$}
    \AxiomC{}%
    \UnaryInfC{#1}%
    \noLine%
    \UnaryInfC{$\vdots$}%
    \noLine%
    \UnaryInfC{#2.}}%

\newcommand{\discharge}[1]{%
	\LeftLabel{$^-\text{\Lightning}\alpha$}%
	\RightLabel{#1}%
}%

\newcommand{\thus}[1]{%
  $\therefore$ #1
}
\newcommand{\biimpl}{\leftrightarrow}
\newcommand{\hsep}{\vspace{1em}\par}
\newcommand{\vsep}{\hspace{0.5em}%
%\vrule%
\hspace{0.5em}}

\begin{document}

    \maketitle

    \section{Inference rules}
    Natural deduction rules augmented with chowder syntax
    \description
    \item{$\smalltriangleright$} acts as n-ary case analysis.
    \item{$\therefore$} acts like the natural deduction mid-line in syntax.
    \par
    \vspace{1em}
    Conjunction:
        \begin{bprooftree}
                \AxiomC{A}
                \AxiomC{B}
            \RightLabel{\thus{$\wedge$I}}
            \BinaryInfC{A $\wedge$ B}
        \end{bprooftree}
        \begin{bprooftree}
                    \AxiomC{A $\wedge$ B}
            \RightLabel{\thus{$\wedge E_L$}}
            \UnaryInfC{A}
        \end{bprooftree}
        \begin{bprooftree}
            \AxiomC{A $\wedge$ B}
            \RightLabel{\thus{$\wedge E_R$}}
            \UnaryInfC{B}
        \end{bprooftree}
\hsep
    Disjunction:
    \begin{bprooftree}
                \AxiomC{A}
        \LeftLabel{}
        \RightLabel{\thus{$\vee I_L$}}
        \UnaryInfC{A $\vee$ B}
    \end{bprooftree}
    \begin{bprooftree}
            \AxiomC{B}
        \RightLabel{\thus{$\vee I_R$}}
        \UnaryInfC{ A $\vee$ B}
    \end{bprooftree}
    \begin{bprooftree}
        \AxiomC{A $\vee$ B}
        \charge{A}{C}
        \charge{B}{C}
    \discharge{\thus{$\vee E$}}
    \TrinaryInfC{C}
\end{bprooftree}

\hsep
Implication:
        \begin{bprooftree}
            \charge{A}{B}
            \discharge{\thus{$\to$I}}
            \UnaryInfC{A $\to$ B}
        \end{bprooftree}
        \begin{bprooftree}
            \AxiomC{$\neg$A}
            \AxiomC{A}
            \RightLabel{\thus{$\neg$E}}
            \BinaryInfC{$\bot$}
    \end{bprooftree}
Negation:
    \begin{bprooftree}
        \charge{A}{$\bot$}
        \discharge{\thus{$\neg$I}}
        \UnaryInfC{$\neg$A}
    \end{bprooftree}
    \begin{bprooftree}
            \AxiomC{A $\to$ B}
            \AxiomC{A}
            \RightLabel{\thus{$\to$E}}
            \BinaryInfC{B}
    \end{bprooftree}

\hsep

Top:
    \begin{bprooftree}
        \AxiomC{}
        \RightLabel{$\top$I}
        \UnaryInfC{$\top$}
    \end{bprooftree}

\hsep

Bot:
    \begin{bprooftree}
        \AxiomC{$\bot$}
        \RightLabel{\thus{$\bot$E}}
        \UnaryInfC{A}
    \end{bprooftree}
\hsep

Bi-implication:

    \begin{bprooftree}
        \charge{A}{B}
        \charge{B}{A}
        \discharge{\thus{$\leftrightarrow$I}}
        \BinaryInfC{A $\biimpl$ B}
    \end{bprooftree}
    \begin{bprooftree}
        \AxiomC{A $\biimpl$ B}
        \AxiomC{A}
        \RightLabel{\thus{$\biimpl{E_L}$}}
        \BinaryInfC{B}
    \end{bprooftree}
    \begin{bprooftree}
        \AxiomC{A $\biimpl$ B}
        \AxiomC{B}
        \RightLabel{\thus{$\biimpl{E_R}$}}
        \BinaryInfC{A}
    \end{bprooftree}
\section{EBNF} 
\begin{Verbatim}
Name ::= [a-zA-Z][a-zA-Z0-9]*
Bindings ::= (Binding ";") Binding?
Binding ::= Name (":" Prop)? "≔" Stmt
Stmt ::= Stmt "‣" Thus | Thus
Thus ::= Thus "∴" PropOpt | Thus "." PropOpt | PropOpt
PropOpt ::= 
()
| Prop

Prop ::= "¬" BinaryProp | BinaryProp
BinaryProp ::=
BinaryProp "∧" Atom
| BinaryProp "∨" Atom
| BinaryProp "→" Atom
| BinaryProp "↔" Atom
| BinaryProp "∧" "¬" Atom
| BinaryProp "∨" "¬" Atom
| BinaryProp "→" "¬" Atom
| BinaryProp "↔" "¬" Atom
| Atom
Atom ::= "T" | "(" Prop ")" | Name
\end{Verbatim}

\section{Some unchecked proofs that parse.}

\begin{Verbatim}
ab_or_cd: (A ∧ B) ∨ (C ∧ D) → B ∨ D
≔ ‣ (A ∧ B) ∨ (C ∧ D)           ;; +⚡1
   ‣ A ∧ B                      ;; +⚡2
    ∴ B                         ;;      ∴ ∧E R
    ∴ B ∨ D.                    ;;      ∴ ∨I L
   ‣ C ∧ D                      ;; +⚡2
    ∴ D                         ;;      ∴ ∧E R
    ∴ B ∨ D.                    ;;      ∴ ∨I R
   ∴ B ∨ D.                     ;; -⚡2 ∴ ∨E
  ∴ (A ∧ B) ∨ (C ∧ D) → B ∨ D.  ;; -⚡1 ∴ →I
  ;
\end{Verbatim}


\begin{Verbatim}
iff_and_or: (A ∧ B) ∨ (C ∧ D) ↔ (B ∧ A) ∨ (D ∧ C)
≔ ‣‣(A ∧ B) ∨ (C ∧ D)                        ;; +⚡1, +⚡2
    ‣ A ∧ B                                  ;; +⚡3
      ∴ B                                    ;; ∴ ∧E R
      ∴ A                                    ;; ∴ ∧E L
      ∴ B ∧ A                                ;; ∴ ∧I
      ∴ (B ∧ A) ∨ (D ∧ C).                   ;; ∴ ∨I L
    ‣ C ∧ D                                  ;; +⚡3
      ∴ D                                    ;; ∴ ∧E R
      ∴ C                                    ;; ∴ ∧E L
      ∴ D ∧ C                                ;; ∴ ∧I
      ∴ (B ∧ A) ∨ (D ∧ C).                   ;; ∴ ∨I L
    ∴ (B ∧ A) ∨ (D ∧ C).                     ;; -3 ∴ ∨E
   ∴ (A ∧ B) ∨ (C ∧ D) → (B ∧ A) ∨ (D ∧ C).  ;; -2 ∴ →I
  ‣‣ (B ∧ A) ∨ (D ∧ C)                       ;; +⚡1, +⚡2
    ‣ B ∧ A                                  ;; +⚡3
     ∴ B ∧ A ∴ A                             ;; ∴ ∧E R
     ∴ B ∧ A ∴ B                             ;; ??, ∴ ∧E L
     ∴ A ∧ B                                 ;; ∴ ∧I
     ∴ (A ∧ B) ∨ (C ∧ D).                    ;; ∴ ∨I L
    ‣ D ∧ C                                  ;; +⚡3
     ∴ C                                     ;; ∴ ∧E R
     ∴ D                                     ;; ??,  ∴ ∧E L
     ∴ C ∧ D                                 ;; ∴ ∧I
     ∴ (A ∧ B) ∨ (C ∧ D).                    ;; ∴ ∨I R
    ∴ (A ∧ B) ∨ (C ∧ D).                     ;; -⚡3 ∴ ∨E
   ∴ (B ∧ A) ∨ (D ∧ C) → (A ∧ B) ∨ (C ∧ D).  ;; -⚡2 ∴ →I
  ∴ (A ∧ B) ∨ (C ∧ D) ↔ (B ∧ A) ∨ (D ∧ C).   ;; -⚡1 ∴ ↔I
  ;
\end{Verbatim}
\end{document}
